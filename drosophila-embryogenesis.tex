\documentclass[11pt,a4paper,twocolumn]{article}
\usepackage[cm]{fullpage}
\usepackage[compact]{titlesec}
\usepackage{soul}
\begin{document}
\title{Establishment of Morphogen Gradients During Drosophila Melanogaster Embryogenesis}
\author{Maxwell Conway}
\date{}
\maketitle
\section{Introduction}
The topic investigated in this project was Drosophila Embryogenesis. This topic, while a highly studied model system, has few stochastic models. Our intention was to build a model based upon values from literature, to avoid conducting a parameters search which could lead to overfitting. We studied development cycles 1-13, with a total of 105 different models, covering the interaction of Bicoid, Caudal, Nanos and Hunchback with 20 different species and 38 reactions. These models varied from those including only Bicoid, which has very well characterized properties, to those including all morphogen gradients, where data was more sparse. We validated our results against quantitative data from the Flyex database \hl{cite}, showing good agreement, and yielding interesting information about the interaction of embryo shape with measurement technique. 

\section{Species Modelled}
\subsection{Bicoid}
Our basic model for Bicoid was a fixed region of mRNA in the anterior of the cell, which was translated to bicoid protein at a constant rate (rates of 0.3, 4 and 7.4 s-1 were used, as several values appeared in the literature). The Bicoid protein had a constant degradation rate of 0.0003s-1. This produced a stable gradient by cycle 13 at all degradation rates. We also tried two refinements of this model
\begin{description}
\item[Increasing Translation rate:]
This model, based on \hl{x}, had the Bicoid mRNA translation rate increasing throughout the duration of development. In our simulations it had not reached stability by cycle 13, but was changing very slowly.
\item[mRNA transport:]
Here we simulated the action of cytoskeletal mRNA transport by starting directed diffusion of Bicoid mRNA towards the posterior of the cell at 1620 seconds into the simulation, but with this mRNA confined to near the cell membrane. Obviously this acted the same as the basic model until the start of diffusion, but after this point produced a flatter gradient, with a peak slightly further from the anterior.
\end{description}

\subsection{Caudal}
Caudal mRNA was created in an even distribution throughout the cytoplasm, with constant protein production and decay rates. However, Bicoid protein could bind to Caudal mRNA to stop transcription, producing a gradient opposite to of Bicoid, peaking in the posterior.

\subsection{Nanos}
Nanos mRNA was created fixed in the posterior of the cell, and created protein in much the same way as Bicoid. However, this protein inhibited Bicoid translation, making the Bicoid gradient steeper in the case where mRNA could diffuse, and causing faster stabilization of the Bicoid gradient in all cases.

\subsection{Hunchback}
The Hunchback model was by far the most complex, but neverthless all rates in it are derived from experimental data. It consists of simple Hunchback transcription and inhibition by Nanos protein, increasing number of nuclei over time (modeled as particles), and 24 reactions to fully model zygotic transcription using 8 different complexes. The simple model of Hunchback inhibition by Nanos produced the expected upward gradient along the anterior-posterior axis, but the full zygotic transcription model suffered performance issues that meant that, even allowing several days of runtime, noise levels were still very high.

\section{Methods}
Some division of labour was used in order to get the best possible result in the time available. My main focus was on the simulation, especially the more technical aspects, and on analysis of results.

\subsection{Framework}
A data handling framework was built around Smoldyn using python and the make build system to allow for many simulations to be run over extended periods. The first stage to this was parameter processing and insertion. This allowed all models to be stored in a spreadsheet, for easy setup and to allow new models to be easily created based on existing ones. A python script processed this spreadsheet to build the Smoldyn files for all of the models listed - far more than could reasonably have been constructed by hand. 

The make build system was used to schedule simulations. This found all models for which the corresponding results file was out of date or not created yet, and scheduled them for simulation, with the number of concurrent simulations equal to the number of processors available, and new simulations scheduled as old ones were completed. Once all results were up to date, they were automatically copied to a folder for analysis, ensuring a consistent results set.

\subsection{Accuracy vs Speed}
Most models required some experimentation to find a point at which the desired level of accuracy could be achieved within a reasonable timeframe. Time step lengths were tried between 1s and 10s, while molecule number scaling factors between 100 and 10000 were used. 

Ensuring simulation accuracy was especially important in simulations of the behaviour of Bicoid mRNA and protein, since for these literature data was available that allowed for accurate parameterization, so that the simulation could be expected to produce quantitatively accurate output. For this reason, these simulations used a 1s timestep to ensure a spatial resolution of 4μm, based on the equation in the Smoldyn manual. A scaling factor of 100 was used, though in the Bicoid models without inhibition effects, this affected only precision, not accuracy.

In the more complex models, such as those including Hunchback, some quantitative accuracy had to be sacrificed for qualitative results, such as by increasing the time step used. Unfortunately, because the Hunchback models took such a disproportionately long time to run, Camgrid was not able to provide a significant speedup, due to Amdahl’s law. A possible extension to this project would be to move to an OpenCL Smoldyn implementation, which could provide a speedup of around 200x \hl{cite}, enabling one to one simulation of molecules within hours or days.

\section{Results}
The best model for Bicoid and Caudal followed the flyex data to within one standard deviation, except for a slight difference in background level. The results 

We are especially happy with this result due to the our mechanisms and parameters being literature derived, so that overfitting is not a major concern.

\nocite{*}
\bibliographystyle{plain}
\bibliography{drosophila-embryogenesis}
\end{document}
